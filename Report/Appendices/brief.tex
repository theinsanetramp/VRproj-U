\chapter{Project Brief}
\lhead{\emph{Project Brief}}
\label{appendix:brief}

\begin{center}
\textbf{Using Data Abstraction and Inter-Frame Interpolation for Low Data Rate Communication Between a 3D Camera and VR Headset} \\
Adam Melvin am20g15@soton.ac.uk \\
Klaus-Peter Zauner kpz@soton.ac.uk
\end{center}

There are many challenges, such as monitoring hostile environments, that call for the use of remotely controlled robots. Often in these scenarios, it would be useful to be able to view the environment with a sense of depth to better understand the scale, and dangers, of the robot’s surroundings. This can be provided through the use of a 3D camera on the robot and Virtual Reality (VR) goggles, however due to the minimum frame rate that can be displayed in a VR headset without causing motion sickness in the user being 60fps (optimally 90fps is preferable), a comfortable and useful experience would often require an unfeasibly high wireless data rate.

The aim of this project is to significantly reduce the data rate required to be transmitted between a robot and teleoperator through the use of data abstraction and inter-frame interpolation. A 3D camera rig mounted on a remotely controlled rover is to take around 10 pictures per second, then they are reduced down to the minimum amount of data required to identify the objects in the environment. These much smaller images are sent wirelessly to a server, where they are used to create a 3D map of the environment. These 3D maps are analysed and intermediate frames are estimated to increase the frame rate from 10fps to 60-90fps. This much higher frame rate estimated map of the environment can then be displayed in the VR headset through the use of a video game engine. Although the estimated frames will not accurately represent the real world, the comfort they provide the user will allow them to focus on the transmitted information without feeling ill.

A simple rover and off-the-shelf VR equipment will be used as a foundation for the system. Different camera systems and interpolation algorithms will be tested on this foundation to discern the setup that produces the best ratio between data rate and usability.
