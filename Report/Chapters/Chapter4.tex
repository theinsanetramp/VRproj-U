\chapter{Plan for Remaining Work}
\lhead{\emph{Plan for Remaining Work}}

When building a system of this scale, great care must be taken to thoroughly test each stage both individually and in the context of the full system. For this reason, the general approach when organising my time was to work from the start point of the process (the hardware of the rover and the data abstraction it uses), making my way through each stage and combining it with the last to create larger subsections of the system. This also allows for better risk management; if a stage does not function as intended to the extent that the project cannot be completed, the work already done would still function as its own smaller system. A gantt chart providing the details of this approach, both going forward and its results, can be found in Appendix E.

A break down of how the budget has been used so far has been presented in Appendix F. The total budget spent is \pounds124.97, leaving \pounds25.03 spare to be used in the event of hardware failure. Also in Appendix F is a full risk assessment for the project going forward, necessary due to the high risk presented by projects of this scale.