\chapter{Conclusion}
\lhead{\emph{Conclusion}}
\label{chapter:conclusion}

The aim of this project was to research the use of data abstraction to minimise the technical issues involved with VR telerobotics through the design and implementation of an abstraction based telerobotics system. Through its use of a 3D model, the system created does not cause motion sickness when operated, successfully mitigating the primary issue with VR telerobotics. The 3D model updates at 20-25 fps with a latency of 0.25-0.5 seconds, providing the user reasonably responsive controls, and is also accurate enough to provide the user spacial awareness. The novel data abstraction algorithm implemented allows this to be achieved with data packets of $<$8kB each.

The next steps for this research would be to replace the R-Pi with a more powerful alternative, with the aim of mitigating the issues with the system that are linked to the R-Pi's bottlenecking, and to improve the system's ability to represent slanted surfaces. As the depth information required to correctly map a slanted surface is already present in the edge detected images of it, it is the depth mapping and depth map filtering that must be redesigned to better extract that information.