\chapter{Project Management}
\lhead{\emph{Project Management}}
\label{chapter:mang}

A comparison of the original project brief (Appendix \ref{appendix:brief}) to the final system reveals some evolution in the scope of this project. The one significant change is the removal of inter-frame interpolation from the design. This decision was made in response to the effectiveness of the system in preventing motion sickness without interpolation, therefore making its addition of no merit. This change is also reflected in the comparison of the work plan produced for the interim report and the actual time frames work was completed in (Appendix \ref{appendix:gantt}). Not only was the inter-frame interpolation component removed, but a complex image pipeline incorporated, shifting the planning considerably. Comparing the shifted timings, the other components were all well planned and were completed roughly within the time frames expected. The risk assessment produced for the interim report was accurate to the risks the project faced, and can be found with comparisons to the actual events that occurred in Appendix \ref{appendix:risk}.

Prior to working on this project, I had some experience of robotics, using Unreal 4 and working with the HTC Vive, and more considerable experience of embedded programming on the Raspberry Pi. To complete a project that is heavily based on these fields therefore required a substantially increase in my competency in them. Furthermore, I had no experience of computer vision, OpenCV, or custom wireless communications, so built my knowledge of these fields completely from scratch for the purposes of this project.